\documentclass{article}

\usepackage{verbatim}
\begin{document}

We want to create a probabilistic programming system that is flexible
and modular. In order to accomplish this, we need to choose the right
abstractions.

A probabilistic program can be thought of as all possible sets of
executions of that program. This can be represented as a tree, where
each node represents a random choice. The distribution specified by a
probabilistic program is the set of its traces weighted according to
the probability of those traces. The normalized weight of each trace
will be equal to the posterior joint probability of all the random
choices made on that trace.

We would like to be able to implement a wide variety of inference
algorithms and various programs. To this end, we need to introduce
some abstractions that unify across these different algorithms. First,
we need the notion of a distribution: we may be able to query a
distribution for various values. For example, we can sample from a
distribution; we can ask for the probability of various elements; we
can sometimes ask for the mean, variance, etc. 

A \textbf{stochastic thunk} is a thunk that uses values drawn from a
distribution. An \textbf{inference procedure} takes a stochastic thunk
and returns a distribution. 

We might want to write the following procedure: 

\begin{verbatim}
(with-mh-inference
 (lambda ()
   (draw-let ((x (draw (pramb (cons 1 0.5)
                              (cons 2 0.5))))
              (y (draw (pramb (cons 1 0.75)
                              (cons 2 0.25)))))
      (draw 
       (with-normal-inference
        (lambda ()
          (draw-let (( k ((+ x y (draw (normal 0 0.2))))))
                    (given k 1.3))))))))
\end{verbatim}

Let's go through this a little bit. \verb+with-mh-inference+ returns
takes a stochastic procedure and returns a distribution. That
procedure sets \verb+x+ and \verb+y+ to values drawn from discrete
distributions over the values \verb+1+ and \verb+2+. \verb+draw-let+
is just like normal \verb+let+ except that it instructs the inference
procedure to record the name of the variable as an element of the
distribution. 

Notice the use of \verb+with-normal-inference+. This inference
procedure knows how to perform inference when the distribution
consists of sums of constants and normal distributions. Note also the
use of \verb+given+: this is the normal conditioning operator.

\end{document}
